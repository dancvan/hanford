%% contrast defect %%

\subsubsection{Simple Michelson}
When handling and preparing your interferometer in a cold state you want to make sure you are minimizing your contrast. We define contrast as the following: 

$$ 
\mathrm{CD} = \frac{{\mid E_{\mathrm{ITMX}}-E_{\mathrm{ITMY}}\mid}^2}{{\mid E_{\mathrm{ITMX}}+E_{\mathrm{ITMY}}\mid}^2}
$$ 
Where $E_\mathrm{ITMX}$ and $E_\mathrm{ITMY}$ are the fields coming from ITMX and ITMY respectively. 

We measured this early to be ?
With the current TCS settings we measure it to be: 


\subsubsection{Locked DRFPMI}
Ideally, at lower arm power you would expect nearly the same mode shapes impinging from the arms onto the beamsplitter after the carrier has traveled through the arms. This is not the case with the finite absorption of the ITMs (bulk and surface) and ETMs (surface) which is guaranteed to introduce contrast with different absorption. Can tune your DARM offset so that you can understand what your contrast is in this configuration?
