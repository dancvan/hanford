%% abstract %%

 The primary role of the currently existing thermal compensation system (TCS) is to probe (via Hartmann wavefront sensors (HWS)) and compensate for (via CO2 lasers and ring heaters) thermo-optic distortions of low spatial frequency that can have potentially adverse affects on mirror geometry which in turn affect interferometer contrast, cavity mode mismatch, reduced cavity buildups, etc. especially when operating at high power. The primary probing is done with Hartmann wavefront sensors placed at each test mass of the dual recycled Fabry-Perot Michelson interferometer and in preparation for O3 we discovered other useful metrics while making final adjustments to TCS actuators. Actuation by negative lensing is performed by a series of installed ring heaters on each of the test masses that form the Fabry Perot cavities while actuation by positive lensing is induced by firing a CO2 laser onto a compensation plate immediately prior to the carrier entering the arm cavity. Relevant measurements and techniques that directly apply to the optimal operation of TCS at the aLIGO Hanford observatory for O3 are detailed. Also discussed are the measurements and effects of point absorbers on input/end test masses. As suggested by models discussed, some of the modeled affects are: higher order mode mismatch from the power recycling cavity to the Y-arm cavity, reduced buildup of RF sidebands in the recycling cavities, and light scatter into higher order modes in the LIGO arm cavities. Keeping the interferometer locked above 30 W input power with the presence of these absorbers is a challenge and may fundamentally limit LHO from achieving a stable configuration at 50 Watts with the currently existing TCS. 